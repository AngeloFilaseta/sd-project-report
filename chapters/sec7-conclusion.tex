\section{Conclusioni}
L'elaborato finale sicuramente soddisfa tutti gli obiettivi imposti nella fase di definizione di progetto, in alcuni casi supera le aspettative, il tutto ponendo la giusta importanza alle varie componenti di gioco e rispettando la scala di priorità definita in una prima fase. \newline

\noindent Al termine di questo progetto possiamo affermare che si è realizzata un'architettura efficiente e a nostro parere ben strutturata, con una particolare attenzione riguardo al lato server per la gestione delle varie operazioni che deve eseguire. \newline

\noindent Ogni servizio svolge il suo compito correttamente garantendo il risultato atteso e si coordina con gli altri senza problemi, in più il corretto funzionamento di ogni servizio è testimoniato dai test creati mediante l'utilizzo di JUnit. \newline

\noindent Al di fuori dell'elaborato, uno dei più grandi goal che ci siamo imposti inizialmente era far si che questo progetto ci permettesse di rimanere in contatto virtuale con amici, che non potremmo vedere a causa delle forti limitazioni alla mobilità dovute alla pandemia in atto. Anche in tal fronte ci possiamo ritenere soddisfatti e confermare di aver ritrovato lo stesso entusiasmo che abbiamo avuto noi nel crearlo durante il gioco.

\subsection{Sviluppi futuri}
Uno sviluppo potrebbe essere l'utilizzo intensivo da parte di numerosi utenti allo scopo di verificare l'efficienza del sistema posto sotto stress e confermarne il corretto funzionamento. \newline

\noindent Altri sviluppi maggiormente estensivi potrebbero vedere un tipo di interazione utente-utente, come scambio di amicizie o creazione di gruppi per velocizzare l'invito in lobby, oppure la personalizzazione di più parametri di gioco da parte dell'admin o di ogni utente, ad esempio la possibilità di scegliere il tempo da lasciare ad ogni utente per produrre una frase o un disegno.\newline

\noindent Per quanto riguarda i report uno degli sviluppi più immediati potrebbe essere l'aggiunta per ogni utente di scegliere se salvare i report o meno ed eventualmente dare la possibilità di eliminare alcuni dei report salvati, qual'ora non siano particolarmente d'interesse per l'utente. \newline

\noindent Ulteriori sviluppi sotto altri fronti vedrebbero una migliore gestione della UI/UX e personalizzazione delle schermate utente, anche potenzialmente con l'integrazione di meccanismi di gamification (come avatar per gli utenti), oppure l'immissione di nuove meccaniche come delle classifiche basate sul voto degli utenti, che potrebbero giudicare le frasi e i disegni migliori.

\subsection{Cosa abbiamo imparato}
Lo sviluppo di questo software ci ha permesso di comprendere al meglio ed imparare numerosi aspetti riguardanti lo sviluppo di sistemi distribuiti, innanzitutto da una prima fase di capire al meglio i ruoli di client e server e i vari modi in cui essi possono comunicare e scambiarsi messaggi, analizzare le varie architetture di rete e i vari collegamenti via socket con i client, presenti su diversi dispositivi e di diversa natura e tecnologie. \newline

\noindent Un altro aspetto molto importante riguarda la programmazione con nuovi framework, servizi e tecnologie che non avevamo avuto modo di conoscere o approfondire in modo esaustivo prima di questo corso. Riteniamo che questo sia il primo progetto in cui si usa un ampio stack di tecnologie differenti tra loro e oggetto del nostro apprendimento è anche stato interfacciare questi vari ambienti allo scopo di far funzionare tutto in maniera coesa come un unico grande sistema distribuito. La mole di questo progetto in più ci ha permesso di comprendere come approcciarsi ad un lavoro potenzialmente oneroso e che domanda organizzazione, comunicazione e attenzione logistica. \newline

\noindent Possiamo concludere che la realizzazione di questo elaborato ci ha permesso di comprendere meglio i7 funzionamento di un sistema distribuito e, unitamente al corso, ci ha fornito conoscenze concrete che riteniamo fondamentali per chi lavora nel nostro ambito e per la comprensione di argomenti che si sviluppano su questi concetti.

\newpage
