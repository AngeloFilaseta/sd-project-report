\section{Analisi dei Requisiti}
\subsection{Obiettivi}
L’idea di progetto consisterebbe nel creare un’architettura Client-Server che permetta lo svolgimento del gioco appena descritto.\newline
\noindent Il server coordinerebbe i giocatori, orchestrandone gli stati, con l’idea di rendere il gioco scalabile. Si vuole rendere fruibile il gioco a client web e mobile, per permettere a più giocatori possibile di giocare, senza necessità che tutti scelgano contemporaneamente la versione web o la versione mobile. Inoltre, vorremmo opzionalmente fornire una chat con la quale i giocatori possano comunicare durante la partita.\newline
Per la prioritizzazione dei nostri obiettivi utilizzeremo il metodo MoSCoW ~\cite{MetodoMo:online}:
\begin{enumerate}
    \item \textit{MUST}:
    \begin{itemize}
        \item Modellazione di tutti i sistemi tramite rappresentazioni di Ingegneria del Software;
        \item Gestione delle stanze lato Server;
        \item Gestione degli stati di gioco lato Server;        \item Creazione di un client Web;
        \item Creazione di Test per verificare il corretto funzionamento del sistema.
    \end{itemize}
    \item \textit{SHOULD}:
    \begin{itemize}
        \item Creazione di un client Mobile;
    \end{itemize}
    \item \textit{WOULD}:
        \begin{itemize}
            \item Implementazione di un sistema di messaggistica;
            \item Possibilità di consultare i report delle partite passate.
        \end{itemize}
    \item \textit{WON'T}:
    \begin{itemize}
        \item Un'interfaccia grafica e curata nel dettaglio e personalizzata.
    \end{itemize}
\end{enumerate}
\noindent Nell'implementazione valuteremo l'utilizzo e l'integrazione di Framework e librerie per semplificare il lavoro e migliorare l'ingegnerizzazione del sistema.\newline
Nello specifico:
\begin{itemize}
    \item Verrà utilizzato il tool-kit Eclipse Vert.X ~\cite{VertX:online} per la realizzazione della parte server. Per il testing verrà principalmente utilizzato JUnit ~\cite{JUnit576:online};
    \item Verrà utilizzato React.js ~\cite{React:online} come base Javascript per la realizzazione della parte client in combinazione con Redux ~\cite{redux:online} per quanto riguarda la gestione degli stati.
    \item Per la parte mobile abbiamo deciso di creare un'app react native che permetta il più possibile il riuso del codice lato client web, reso responsive utilizzando Bootstrap.
    \item Verranno utilizzati Gradle per la build automation lato server  ~\cite{GradleEnterprise:online} e npm ~\cite{npm:online} per la gestione dei pacchetti lato Javascript.
\end{itemize}
\subsection{Expected Deliverables}
Il gruppo si impegna a consegnare:
\begin{itemize}
    \item la struttura di un database relazionale in cui verranno memorizzati i dati degli utenti;
    \item i sorgenti dello stateful server;
    \item i sorgenti dei relativi client (web e mobile).
\end{itemize}
\subsection{Divisione del Lavoro}
Tenendo conto delle attuali restrizioni alla mobilità, la divisione del lavoro verrà valutata dinamicamente utilizzando strumenti di comunicazione telematica.\newline
In linea di massima però, è stata definita una scaletta alla quale cercheremo di attenerci:
\begin{enumerate}
    \item La fase di progettazione verrà svolta congiuntamente da tutti i membri del gruppo, al fine di una comprensione coesa dell'architettura del progetto;
    \item La definizione della struttura del Database verrà anch'essa svolta congiuntamente, mentre la sua implementazione verrà affidata ad un solo membro del gruppo che, dopo averla portata a termine si unirà alla prossima fase;
    \item Durante l'implementazione del DB, gli altri membri del gruppo inizieranno a creare lo scheletro del server, creando le basi per poter parallelizzare il lavoro;
    \item Una volta raggiunta una prima versione stabile del server inizierà la fase di testing di quest'ultimo in parallelo all'implementazione dei client Web e Mobile.
\end{enumerate}

\noindent Riassumendo, gli obiettivi di sviluppo per questo progetto sono:

\begin{itemize}
    \item la realizzazione di un gioco distribuito, multipiattaforma, che ricorre all'utilizzo delle WebSocket;
    \item realizzazione del software sfruttando un'architettura a Microservizi;
    \item comprensione delle tecnologie WebSocket e Vert.X
    \item sviluppo di un applicazione Java, con supporto Gradle ed implementazione del framework Vert.X, allo scopo di ottenere un'architettura modulare e scalabile;
    \item sviluppo di un client web che possa connettersi al server e permettere all'utente di interagire con altri tramite dinamiche di gioco e chat in tempo reale;
    \item sviluppo di un client mobile in react native, che consenta la portabilità del client web reso responsive su piattaforme mobili;
    \item sviluppo di uno storage tramite database in MySql che consenta la persistenza dei dati inseriti e generati;
    \item gestione dei ruoli degli utenti (Admin e User) che prendono parte al gioco;
    \item gestione delle lobby di gioco, sia pubbliche che private, e con accesso tramite codice (nel caso delle lobby private) e tramite scelta della lingua (nel caso di lobby pubbliche).
\end{itemize}
\newpage
