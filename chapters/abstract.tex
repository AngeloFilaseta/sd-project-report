\begin{abstract}
\noindent Il progetto nasce dall’idea di digitalizzare un gioco, una versione modificata di "sigaretta" ~\cite{wiki:Sigarettagioco} nato per caso dal gruppo di amici di uno dei nostri componenti, e renderlo digitale al fine di accorciare le distanze in questo periodo di forti limitazioni alla mobilità.\newline

\noindent Il gioco, nella sua versione fisica, si svolge nel seguente modo: partecipano \textit{n} giocatori (con \textit{n} strettamente maggiore di 2, senza un limite superiore) ognuno dotato di un foglio di carta. Inizialmente ogni partecipante scrive una frase a piacimento su tale foglio e, senza farsi vedere dagli altri partecipanti, piega il foglio di modo che la frase non sia visibile.\newline

\noindent Una volta che ogni giocatore ha scritto la frase, passa il foglio al giocatore alla sua sinistra che, letta la frase, provvederà a fare un disegno che la rappresenti. Analogamente a prima poi i giocatori piegheranno il foglio di modo da nascondere il disegno e lo passeranno nuovamente al giocatore alla loro sinistra che invece stavolta (potendo vedere solo il disegno) scriverà una frase che lo descriva.\newline

\noindent Viene seguito questo pattern finché, dipendentemente dalla scelta dei giocatori, si è raggiunto un numero di “passaggi” o il giro è terminato. Alla fine del gioco vengono mostrati a tutti i giocatori i fogli nella loro interezza che, necessariamente, avranno preso una deriva strana rispetto alla frase scritta inizialmente. Non vi è dunque un vincitore, il gioco difatti punta solo ad analizzare i risultati finali.
\end{abstract}
\newpage
